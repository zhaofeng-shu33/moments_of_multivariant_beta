\documentclass[mathpazo]{cmr}
\usepackage{amssymb}
\usepackage{amsmath}
\usepackage{amsthm}
\usepackage{mathtools}
\DeclarePairedDelimiter\abs{\lvert}{\rvert}
\def\E{\mathbb{E}}
\def\Beta{\textrm{Beta}}

%%%%% author macros %%%%%%%%%
% place your own macros HERE
%%%%% end %%%%%%%%%

\begin{document}
%%%%% title : short title may not be used but TITLE is required.
% \title{TITLE}
% \title[short title]{TITLE}
\title{Second-order Matrix Extension of Beta Distribution and its high Order Moments}

%%%%% author(s) :
% single author:
% \author[name in running head]{AUTHOR\corrauth}
% [name in running head] is NOT OPTIONAL, it is a MUST.
% Use \corrauth to indicate the corresponding author.
% Use \email to provide email address of author.
% \footnote and \thanks are not used in the heading section.
% Another acknowlegments/support of grants, state in Acknowledgments section
% \section*{Acknowledgments}
\author[Feng Zhao et~al.]{Feng Zhao\affil{1}\comma\corrauth and Shao-Lun Huang \affil{2}}
\address{\affilnum{1}\ Department of Electronics Engineering, Tsinghua University, Beijing, 100089, P.R. China \\
\affilnum{2}\
Tsinghua-Berkeley Shenzhen Institute, Shenzhen, 518000
}
\emails{ {\tt zhaof17@mails.tsinghua.edu.cn} (Feng Zhao), {\tt shaolun.huang@sz.tsinghua.edu.cn}(Shao-Lun Huang)}

% multiple authors:
% Note the use of \affil and \affilnum to link names and addresses.
% The author for correspondence is marked by \corrauth.
% use \emails to provide email addresses of authors
% e.g. below example has 3 authors, first author is also the corresponding
%      author, author 1 and 3 having the same address.
% \author[Z. Zhang et~al.]{Zhengru Zhang\affil{1}\comma\corrauth,
%       Author Chan\affil{2}~and Author Zhao\affil{1}}
% \address{\affilnum{1}\ School of Mathematical Sciences,
%          Beijing Normal University,
%          Beijing 100875, P.R. China. \\
%           \affilnum{2}\ Department of Mathematics,
%           Hong Kong Baptist University, Hong Kong SAR.}
% \emails{{\tt zhang@email} (Z.~Zhang), {\tt chan@email} (A.~Chan),
%          {\tt zhao@email} (A.~Zhao)}
% \footnote and \thanks are not used in the heading section.
% Another acknowlegments/support of grants, state in Acknowledgments section
% \section*{Acknowledgments}


%%%%% Begin Abstract %%%%%%%%%%%
\begin{abstract}
In this article, we consider a second-order matrix extension of Beta distribution.
That is a distribution on second-order random matrix.
We will give the analytical
formula for its high order moments, which is superior over general
numerical integration method.
\end{abstract}
%%%%% end %%%%%%%%%%%

%%%%% AMS/PACs/Keywords %%%%%%%%%%%
%\pac{}
\ams{62H10, 33B15%The information of the AMS subject classification can be found in http://mathscinet.ams.org/msc/msc2010.html
}
\keywords{multivariate Beta distribution, higher order moments, multivariate Gamma function.}

%%%% maketitle %%%%%
\maketitle


%%%% Start %%%%%%
\section{Introduction}
David introduces an extension of
multivariate extension for Beta distribution,
denoted as $\mathbf{B}(\alpha, \beta; I_p)$ (see \cite{david1981}) .
It is a random $p\times p$  symmetric matrix $W$ whose density
function is given by
\begin{align}
p(W) &= \frac{1}{B_p(\alpha, \beta)}\abs{I-W}^{\alpha-\frac{p+1}{2}}
\abs{W}^{\beta-\frac{p+1}{2}} \textrm{ where } W, I-W \in S_{p,p}^{++}
\label{eq:distr}\\
B_p(\alpha, \beta) &= \int_{W,I-W \in S_{p,p}^{+} }\abs{I-W}^{\alpha-\frac{p+1}{2}}
\abs{X}^{\beta-\frac{p+1}{2}}dX \textrm{ where } \alpha, \beta > \frac{p-1}{2}
\end{align}
$B_p(\alpha, \beta)$ is called the multivariate Beta function (see \cite{siegel_1935}); 
$\abs{W}$ is the determinant of matrix $W$ and $S_{p,p}^{++}$ is the 
collection of positive
definite matrix.
When $p=1$, the distribution reduces to normal Beta distribution for
$0<x<1$.

This extension may have useful applications in multivariate statistical
problems but little is known about the analytical property of such extension.

Konno has derived the formula of moment up to second order (see \cite{konno_1988}).
In this paper, we focus on the case $p=2$ and deduce the analytical form of 
higher order moments for $\mathbf{B}(\alpha, \beta; I_2)$.
This formula
includes the expectation and variance, which are the first and second
order moment respectively. The higher order moments formula, as
far as we know, is novel and can be used directly in the computation
related with multivariate Beta models instead of approximate
numerical integration.

In this article, the following notation convention is adopted:
$W=\begin{pmatrix} X & Z \\ Z & Y \end{pmatrix}$ is the symmetric random
matrix to be considered. Its distribution is given by Equation \eqref{eq:distr}.
$\abs{W}=XY-Z^2$.
Let $\E_{\alpha,\beta}[f(X,Y, Z)] = \int f(X,Y,Z)p(W)dW$ denotes the expectation
with $\mathbf{B}(\alpha, \beta;I_2)$ where $f(X, Y, Z)$ is an arbitrary function with three
variables. We will compute $\E_{\alpha,\beta}[f(X,Y, Z)]$
when $f(X,Y,Z)$ takes the monomial form: $f(X,Y,Z)=X^m Y^r Z^{2t}$.

\section{Marginal Distribution}
In this section we will compute $\E_{\alpha,\beta}[f(X,Y, Z)]$
for $f(X,Y,Z)=X^m$. That is, we consider the marginal distribution for $X$,
which turns out to be one dimensional Beta distribution.
To accomplish this goal, we need the following lemma:
\begin{lemma}\label{lem:AB}
Let $A = XY - Z^2, B = 1 - X - Y + A$, then we have
\begin{align}
\E_{\alpha, \beta}[Af(X,Y,Z)] =&
\frac{\alpha(\alpha-1/2)}{(\alpha+\beta)(\alpha+\beta-1/2)}\E_{\alpha+1, \beta}[f(X,Y,Z)]
\label{eq:Aexp} \\
\E_{\alpha,\beta}[Bf(X,Y,Z)] =&
\frac{\beta(\beta-1/2)}{(\alpha+\beta)(\alpha+\beta-1/2)}\E_{\alpha, \beta+1}[f(X,Y,Z)]
\label{eq:Bexp}
\end{align}
\end{lemma}
\begin{proof}
For multivariate Beta function we have
$B_p(a, b) = \frac{\Gamma_p(a)\Gamma_p(b)}{\Gamma_p(a+b)}$
where $\Gamma_p$ is the multivariate Gamma function (see \cite{ingham_1933}).
For $p=2$ we have $\Gamma_2(a) = \sqrt{\pi}\Gamma(a)\Gamma(a-1/2)$.
\begin{align*}
\frac{\E_{\alpha, \beta}[Af(X,Y,Z)]}{\E_{\alpha+1, \beta}[f(X,Y,Z)]} &
=\frac{B_2(\alpha+1,\beta)}{B_2(\alpha,\beta)}\\
&=\frac{\Gamma_2(\alpha+1)}{\Gamma_2(\alpha)}
\frac{\Gamma_2(\alpha+\beta)}{\Gamma_2(\alpha+\beta+1)}\\
& =\frac{\Gamma(\alpha+1)}{\Gamma(\alpha)}
\frac{\Gamma(\alpha+1/2)}{\Gamma(\alpha-1/2)}
\frac{\Gamma(\alpha+\beta)}{\Gamma(\alpha+\beta+1)}
\frac{\Gamma(\alpha+\beta-1/2)}{\Gamma(\alpha+\beta+1/2)}\\
&=\frac{\alpha(\alpha-1/2)}{(\alpha+\beta)(\alpha+\beta-1/2)}
\end{align*}
Thus Equation \eqref{eq:Aexp} is proved and Equation \eqref{eq:Bexp} follows similarly.
\end{proof}
Using the above Lemma, we give the main conclusion of this section:
\begin{theorem}\label{thm:Xm}
$\E_{\alpha, \beta}[X^m] =
\prod_{i=0}^{m-1}\frac{\alpha+i}{\alpha+\beta+i}$, thus $X$
conforms to Beta distribution $\Beta(\alpha, \beta)$.
\end{theorem}
\begin{proof}
Since the position of $X$ and $Y$ is symmetric,
$\E_{\alpha, \beta}[X]=\E_{\alpha, \beta}[Y]$.
Taking the expectation about $\Beta_2(\alpha, \beta)$
on both sides of $B=1-X-Y+A$ and using the
conclusion of Lemma \ref{lem:AB} we have
\begin{equation*}
\frac{\beta(\beta-1/2)}{(\alpha+\beta)(\alpha+\beta-1/2)}
= 1 - 2\E_{\alpha, \beta}[X] +
\frac{\alpha(\alpha-1/2)}{(\alpha+\beta)(\alpha+\beta-1/2)}
\end{equation*}
Solving the about equation we get
$\E_{\alpha, \beta}[X]=\frac{\alpha}{\alpha + \beta}$.
Recursively using Equation \eqref{eq:Aexp} with $f(X,Y,Z)=X$
we have $\E_{\alpha, \beta}[X^m] =
\prod_{i=0}^{m-1}\frac{\alpha+i}{\alpha+\beta+i}$.
This expression of higher order moments
is the same with that of Beta distribution and by
the uniqueness of moment generating function, we
conclude that $X$ is actually Beta distribution $B(\alpha,
\beta)$.
\end{proof}
\section{Mixed Moment}
In this section, we further compute $\E_{\alpha,\beta}[X^mY^rZ^{2t}]$.
By symmetric property  $\E_{\alpha,\beta}[X^mY^rZ^{2t+1}] = 0$.
Therefore we only need to consider the case when the power of $Z$
is even.
\begin{theorem}\label{thm:mm}
\begin{equation}\label{eq:ZXtm}
\E_{\alpha, \beta}[X^mZ^{2t}] = \frac{(2t-1)!!}{2^t}
\prod_{i=0}^{t-1}
\frac{\beta+i}{(\alpha+\beta+i-1/2)}
\frac{\prod_{i=0}^{t+m-1}\alpha+i}{\prod_{i=0}^{2t+m-1}
\alpha+\beta+i}
\end{equation}
\end{theorem}
\begin{proof}
We use induction to show Equation \eqref{eq:ZXtm} is true.
Firstly, Equation \eqref{eq:ZXtm} is true for $t=0$ from 
Theorem \ref{thm:Xm}. Let $A, B$ be the same as those
in Lemma \ref{lem:AB}.
Suppose Equation \eqref{eq:ZXtm} holds for $\E[Z^{2t-2}X^m]$,
using $Z^2=XY-A=X(1-X-A+B)-A$, then
\begin{align*}
\E_{\alpha,\beta}[Z^{2t}X^m] &=
\E_{\alpha,\beta}[Z^{2t-2}X^m(X-X^2+AX-BX-A)] \\
&= \E_{\alpha,\beta}[Z^{2t-2}(X^{m+1} - X^{m+2})] +
\E_{\alpha,\beta}[A Z^{2t-2} (X^{m+1} - X^m)] \\
&-
\E_{\alpha,\beta}[BZ^{2t-2} X^{m+1}] \\
&= \left(1-\frac{\alpha+t+m}{\alpha+\beta+2t+m-1}
\right)\E_{\alpha, \beta}[Z^{2t-2}X^{m+1}] \\
&+
\frac{\alpha(\alpha-1/2)}{(\alpha+\beta)(\alpha+\beta-1/2)}
\E_{\alpha+1,\beta}[Z^{2t-2}(X^{m+1} - X^m)] \\
&-\frac{\beta(\beta-1/2)}{(\alpha+\beta)(\alpha+\beta-1/2)}
\E_{\alpha,\beta+1}[Z^{2t-2}X^{m+1}]\\
&=\frac{\beta+t-1}{\alpha+\beta + 2t+m-1}
\E_{\alpha,\beta}[Z^{2t-2}X^{m+1}] \\
&+\frac{(\alpha+t+m)(\alpha-1/2)}{(\alpha+\beta+2t-1+m)(\alpha+\beta + t - 3/2)}\\
&\cdot
\left(1-\frac{\alpha+\beta+2(t-1)+m+1}{\alpha+t+m}\right)
\E_{\alpha, \beta}[Z^{2t-2}X^{m+1}] \\
&-\frac{(\beta+t-1)(\beta-1/2)}{(\alpha+\beta+2t-1+m)(\alpha+\beta + t - 3/2)}
\E_{\alpha,\beta}[Z^{2t-2}X^{m+1}]\\
&=\frac{(t-1/2)(\beta+t-1)}{(\alpha+\beta+t-3/2)(\alpha+\beta+2t+m-1)}
\E_{\alpha, \beta}[Z^{2t-2}X^{m+1}]
\end{align*}
Using Equation \eqref{eq:ZXtm} for $t-1$
we can get the same form of expression for $t$.
\end{proof}
From Theorem \ref{thm:mm},
we can get
the general formula for the mixed moment when $m\geq r$:
\begin{corollary}\label{cor:mr}
\begin{align}
\E_{\alpha, \beta}[X^mY^rZ^{2t}] &= \frac{(2t-1)!!}{2^t}
\frac{\prod_{j=0}^{t-1} (\beta+j)}{\prod_{j=0}^{t+r-1} (\alpha+\beta-1/2+j)}
\frac{\prod_{j=0}^{t+m-1}(\alpha+j)}{\prod_{j=0}^{2t+m-1} (\alpha+\beta+j)}\notag\\
&\cdot\sum_{i=0}^r \frac{1}{2^i}\binom{r}{i}\prod_{j=1}^i (2t-1+2j)
\frac{\displaystyle\prod_{j=0}^{r-i-1}(\alpha-1/2+j)
\prod_{j=0}^{i-1}(\beta+j+t)}{\prod_{j=0}^{i-1}(\alpha+\beta+j+2t+m)}
\label{eq:mrt}
\end{align}
\end{corollary}
Since $\E_{\alpha, \beta}[X^mY^rZ^{2t}]=\E_{\alpha, \beta}[X^rY^mZ^{2t}]$, 
when $m<r$ we can exchange $m$ and $r$ and use Corollary \ref{cor:mr}.
\begin{proof}
From Lemma \ref{lem:AB}, we have
$X^m Y^r Z^{2t} = X^{m-r}(A+Z^2)^r Z^{2t} $.
Using binomial theorem we have
$\E_{\alpha, \beta}[X^m Y^r Z^{2t}] = \sum_{i=0}^r \binom{r}{i}\E_{\alpha, \beta}
[A^{r-i}X^{m-r}Z^{2(t+i)}]$. Then using Equation \eqref{eq:Aexp} recursively
we have 
\begin{align*}
\E_{\alpha, \beta}
[A^{r-i}X^{m-r}Z^{2(t+i)}] =&
\prod_{j=0}^{r-i-1}\frac{(\alpha+j)(\alpha+j-1/2)}
{(\alpha+j+\beta)(\alpha+j+\beta-1/2)} \\
\cdot & \E_{\alpha+r-i, \beta}
[X^{m-r}Z^{2(t+i)}]
\end{align*}
Using Theorem \ref{thm:mm} we can finally get the expression in
Equation \eqref{eq:mrt}.
\end{proof}
\section{Conclusion}
We have derived the formula of high order moments for multivariate Beta distribution
in second-order matrix case. This result is helpful when the computation of moments
are required in some statistics problem.

%%%% Bibliography  %%%%%%%%%%
\begin{thebibliography}{99}
\bibitem{david1981} Dawid, A.~P. Some matrix-variate distribution theory: Notational
  considerations and a bayesian application.{\it Biometrika} 1981, {\bf 68}(1): 265-274.

\bibitem{siegel_1935} Siegel, C.~L. Über die analytische theorie der quadratischen formen((in German).
  {\it Anal. Math.} 1935, {\bf 36}(3): 527-606.

\bibitem{konno_1988} Konno, Y. Exact moments of the multivariate f and beta distributions.
{\it J. Japan Stat. Soc.} 1988, {\bf 18}(2): 123-130

\bibitem{ingham_1933} Ingham, A.~E. An integral which occurs in statistics. {\it Math. Proc. Cambridge Philos. Soc.}, 1933, {\bf 29}(2): 271 - 276.
\end{thebibliography}

\end{document}
