%%%%%%%%%%%%%%%%%%% CMR Begin %%%%%%%%%%%%%%%%%
\documentclass[twoside]{article}%{cmr}%                       %         %
%\special{pdf: pagesize width 19.1cm height 26.6cm}           %         %
\textwidth=14truecm  \textheight=20.5truecm                   %         %
\oddsidemargin=0mm   \evensidemargin=0mm                      %         %
\usepackage{amssymb,amsmath,epsfig,mathrsfs}                  %  Don't  %
%\input cchead.sty                                             %         %
\pagestyle{headings}                                          %         %
\headheight=10mm\headsep=3mm\topmargin=0mm                    %         %

\usepackage{mathtools}
\DeclarePairedDelimiter\abs{\lvert}{\rvert}
\def\E{\mathbb{E}}
\def\Beta{\textrm{Beta}}
                                                              %         %
\begin{document}                                              %         %
                                                              % delete  %
\catcode`!=11                                                 %         %
\let\!int\int \def\int{\displaystyle\!int}                    %         %
%%\let\!frac\frac \def\frac{\displaystyle\!frac}              %         %
\catcode`!=12                                                 %         %
                                                              %   or    %
\def\pd#1#2{\frac{\partial#1}{\partial#2}}                    %         %
\baselineskip=5truemm \font\abstractsizebf=cmbx9              %         %
\font\abstractsizeit=cmti9 \font\shumeiit=cmti8               %  alter  %
\font\abstractsize=cmr9 \font\shumeisize=cmr8                 %         %
\abovedisplayskip=.5mm plus .5mm minus .5mm                   %         %
\belowdisplayskip=.5mm plus .5mm minus .5mm                   %         %
\def\evenhead{\pushziti\hbox to                               %  this   %
\hsize{\rm\shumeisize\thepage\hfill                           %         %
    \shumeisize  \hspace{10mm}COMM. MATH. RES.                %  part   %
    \hfill VOL.~3?}\popziti}                                  %         %
\def\oddhead{\hbox to \hsize{\shumeisize NO.~?\hfill          %         %
    {\shumeiit XXXX X. X. et al.}\ \
%将XXXX X. X.换成第一作者姓、名的缩写,字母全部大写 change to the abbreviated name of first author %%
    \shumeisize A THEOREM OF NEHARI TYPE
%将A THEOREM OF NEHARI TYPE换成该文标题缩写(不是单词缩写),字母全部大写.change to the abbreviated title of your paper %%
    \hfill\rm\shumeisize\thepage}}                            %         %
\title{\vspace{-1in}                                          %         %
   \parbox{\linewidth}{\footnotesize\noindent                 %         %
     \unitlength=1mm                                          %  Don't  %
     \doublerulesep0.5pt                                      %         %
%     \begin{picture}(28,0)                                    %         %
%     \put(0,0){\line(1,0){28}}                                % delete  %
%     \put(0,-0.15){\line(1,0){28}}                            %         %
%     \put(0,-0.3){\line(1,0){28}}                             %   or    %
%     \put(0,-1){\line(1,0){28}}                               %         %
%     \put(0,-15.5){\line(1,0){28}}                            %  alter  %
%     \put(0,-4){\sc Communications}                           %         %
%     \put(0,-6.2){\sc in}                                     %         %
%     \put(0,-9){\sc Mathematical}                             %  this   %
%     \put(0,-11.5){\sc Research}                              %         %
%     \put(0,-14){\bf3?\rm(?)(201?), ???--???}                 %  part   %
%     \end{picture}                                            %         %
}                                                             %         %
   \vspace{10mm}%%{\bigskipamount}                            %         %
\\                                                            %         %
\bf Second-order Matrix Extension of Beta Distribution and its high Order Moments                                         %题目 change to your title %%
\footnotetext{                                                %         %
\bf\scriptsize Received date: \rm\scriptsize Jan. 26, 2020.   %投稿日期 %
\hfil\break\indent                                            %         %
\bf\scriptsize\hspace{1.3mm} Foundation item: \rm\scriptsize  %         %
The NSF (10671028) of China.                                  %基金 change to your supported foundation %%
\hfil\break\indent                                            %         %
\bf\scriptsize\hspace{2.3mm}{\rm *} {Corresponding author{\rm .}} %         %
\hfil\break\indent                                            %         %
\bf\scriptsize\hspace{2.3mm}{E-mail address: \rm\scriptsize   %         %
Zhouyifeng@jlu.edu.cn (Zhou Y F), Yangsi@jlu.edu.cn (Yang S).}%%电子邮件 change to your and the corresponding author's E-mail address %%
}}
\date{}
\author{\abstractsize\sc Zhao Feng$^{1, {\text *}}$,~ %(周一峰)    %所有作者的中文姓名必须填写
and~ Huang Shao-Lun$^2$ %(张三)                                       %如有通讯作者(非第一作者),请在其名字上加注"^{\text*}",并
%and~ Yang Si$^{1, {\text *}}$   %(杨四)                       %在第一作者的E-mail后给出其E-mail。change to authors' full names%          
\\
\abstractsize{({\it\abstractsizeit 1. Department of Electronics Engineering, Tsinghua University, Beijing, 100089})}\\  %所有作者单位
\abstractsize{({\it\abstractsizeit 2. Tsinghua-Berkeley Shenzhen Institute, Shenzhen, 518000})}
%\\[5mm]Communicated by  编委
}

\maketitle

%% begin of abstract etc.

\let\oldparindent\parindent
\leftskip10truemm
\rightskip10truemm
{
%\scriptsize
\parindent=0pt
\abstractsizebf Abstract: \rm
\abstractsize
In this article, we consider a second-order matrix extension of Beta distribution.
That is a distribution on second-order random matrix.
We will give the analytical
formula for its high order moments, which is superior over general
numerical integration method.                                                    %% 摘要change to abstract %%
\\
\abstractsizebf Key words:
\abstractsize  multivariate Beta distribution, higher order moments, multivariate Gamma function                   %% 关键词change to keywords %%
\\
\abstractsizebf 2010 MR subject classification:
\abstractsize 62H10 , 33B15                                     %% 分类号change to 2000 MR subject classification %%
\\
\abstractsizebf Document code:
\abstractsize A
\\
\abstractsizebf Article ID:
\abstractsize undetermined
\\
\abstractsizebf DOI:
\abstractsize undetermined
}

\def\parindent{\oldparindent}
\leftskip0truemm
\rightskip0truemm

%% end of abstract

\newtheorem{theorem}{Theorem}[section]                                               %
\newtheorem{definition}{Definition}[section]                                         %
\newtheorem{lemma}{Lemma}[section]                                                   %
\newtheorem{proposition}{Proposition}[section]                                       %
\newtheorem{corollary}{Corollary}[section]                                           %
\newtheorem{remark}{Remark}[section]                                                 %
\renewcommand{\theequation}{\thesection.\arabic{equation}}                           %
\catcode`@=11                                                                        %
\@addtoreset{equation}{section}                                                      %
\setcounter{page}{1}                                                                 %
\thispagestyle{empty}                                                                %
\catcode`@=12                                                                        %
%%%%%%%%%%%%%%%%%%%%% CMR End  %%%%%%%%%%%%%%%%%%%%%%%%%%%%%%%%%%%%%%%%%%%%%%%%%

\section{Introduction}
%%行文中引用参考文献的格式如下,引用时一律按行文中引用的先后为序;没有引用的参考文献不得列入文献表。

\cite{david1981} introduces an extension of
multivariate extension for Beta distribution,
denoted as $\mathbf{B}(\alpha, \beta; I_p)$.
It is a random $p\times p$  symmetric matrix $W$ whose density
function is given by
\begin{align}
p(W) &= \frac{1}{B_p(\alpha, \beta)}\abs{I-W}^{\alpha-\frac{p+1}{2}}
\abs{W}^{\beta-\frac{p+1}{2}} \textrm{ where } W, I-W \in S_{p,p}^{++}
\label{eq:distr}\\
B_p(\alpha, \beta) &= \int_{W,I-W \in S_{p,p}^{+} }\abs{I-W}^{\alpha-\frac{p+1}{2}}
\abs{X}^{\beta-\frac{p+1}{2}}dX \textrm{ where } \alpha, \beta > \frac{p-1}{2}
\end{align}
$B_p(\alpha, \beta)$ is called the multivariate Beta function (\cite{siegel_1935}); 
$\abs{W}$ is the determinant of matrix $W$ and $S_{p,p}^{++}$ is the 
collection of positive
definite matrix.
When $p=1$, the distribution reduces to normal Beta distribution for
$0<x<1$.

This extension may have useful applications in multivariate statistical
problems but little is known about the analytical property of such extension.

\cite{konno_1988}
has derived the formula of moment up to second order.
In this paper, we focus on the case $p=2$ and deduce the analytical form of 
higher order moments for $\mathbf{B}(\alpha, \beta; I_2)$.
This formula
includes the expectation and variance, which are the first and second
order moment respectively. The higher order moments formula, as
far as we know, is novel and can be used directly in the computation
related with multivariate Beta models instead of approximate
numerical integration.

In this article, the following notation convention is adopted:
$W=\begin{pmatrix} X & Z \\ Z & Y \end{pmatrix}$ is the symmetric random
matrix to be considered. Its distribution is given by Equation \eqref{eq:distr}.
$\abs{W}=XY-Z^2$.
Let $\E_{\alpha,\beta}[f(X,Y, Z)] = \int f(X,Y,Z)p(W)dW$ denotes the expectation
with $\mathbf{B}(\alpha, \beta;I_2)$ where $f(X, Y, Z)$ is an arbitrary function with three
variables. We will compute $\E_{\alpha,\beta}[f(X,Y, Z)]$
when $f(X,Y,Z)$ takes the monomial form: $f(X,Y,Z)=X^m Y^r Z^{2t}$.

\section{Definition, Theorem, Lemma, Remark and Proof} %%2

\begin{definition} \quad %%2.1
abcd
\end{definition}

\begin{proposition}{\hspace{-1.5mm}}$^{[3]}$ \quad %%2.1
abcd
\end{proposition}

\begin{theorem} \quad %%2.1
abcd
\end{theorem}

\noindent{\it Proof.} \quad
abcd

\begin{remark} \quad %
{\rm % Remark 用正体
abcd
}
\end{remark}

\section{Photographs and Tables}%%The numbers and denotations in figures MUST be smaller than those in the text. %%
                                %%图和表中的数字、符号、公式等必须与行文的字体一致,字号应略小于行文             %%

%\begin{figure}[h]
%\centerline{\psfig{file=cmrf1.eps,width=5cm}}
%\end{figure}
\vspace*{-8pt}
\begin{center}
{\small {\bf Fig. 3.1} A schematic illustration of dissociative recombination. The
direct mechanism, 4m$^2_\pi$ is initiated when the
molecular ion captures an electron with kinetic energy.}
\end{center}
Figures are to be inserted in the text nearest their first reference.
If the author requires the publisher to reduce
the figures, ensure that the figures (including letterings and
numbers) are large enough to be clearly seen after reduction. If
photographs are to be used, only black and white ones are acceptable.

Figures are to be sequentially numbered in Arabic numerals. The
caption must be placed below the figure.


%\begin{center}
%\epsfig{width=9cm,file=***.eps}                          %% .eps 其他插图格式参考
%\begin{minipage}{11cm}
%\begin{center}
%{\small {\bf Fig. 3.2} Figure's name}
%\end{center}
%\end{minipage}
%\end{center}

%\setlength{\unitlength}{1mm}
%\begin{center}\begin{picture}(35,50)
%\put(-12,50){\special{em:graph ***.bmp x=5cm y=5cm}}     %% .bmp 其他插图格式参考
%\end{picture}
%\end{center}
%\begin{center}
%{\small {\bf Fig. 3.3} Figure's name}
%\end{center}

%%%%%%%%%%%%%%%%%%%%%%%%%%%%%%%%%%%%%%%%%%表格
\begin{center}
{\small {\bf Table 3.1} \quad The test problems set  \\    %表格题目 change to your tables %%
\vspace{1mm}
\doublerulesep 0.5pt
\begin{tabular}{ccc}
\hline\hline\\
[-3mm]
 Problem number & Problem Name & Dimension  \\
[.8mm]
 \hline
 \\[-3mm]
       1    &Helical Valley             &3    \\
       2    &Biggs EXP6                 &6    \\
       3    &Gaussian                   &3    \\
       4    &Variably. Dimensioned.     &10   \\
       5    &Watson                     &12   \\
       6    &Penalty I                  &10   \\
       7    &Penalty II                 &10   \\
       8    &Brown. Dennis.             &4    \\
       9    &Gulf R. D.                 &3    \\
       10   &Trigonometric              &20   \\
       11   &Extended. Rosenbrock.      &10   \\
       12   &Extended. Pow. Singular.   &4    \\
       13   &Beale                      &2    \\
       14   &Wood                       &4    \\
       15   &Chebyquad                  &8    \\
       16   &Scaled Cube ($c=10^2$)     &2    \\
[.8mm]
\hline\hline
\end{tabular}}
\end{center}
Tables should be inserted in the text as close to the point of
reference as possible. Some space should be left above and below
the table.

Tables should be numbered sequentially in the text.
Captions are to be centralized above the tables.


\begin{thebibliography}{stringX}
%%参考文献 All references should be listed according to the order cited in the text and bracketed.%%
\leftskip=-8mm
\parskip=-1mm
\small

\bibitem{david1981} Surname(s) Initial(s). {P}aper title. {\it abbreviated title of the journal},
year of publication, {\bf volume number}(issue number): relevant numbers. %%Papers
%参考文献[1]Papers:作者姓(首字母大写) 名的首字母缩写. 文章名. {\it 杂志名缩写}, 出版年,{\bf 卷}(期): 起止页码.

\bibitem{david1981} Dawid, A.~P. Some matrix-variate distribution theory: Notational
  considerations and a bayesian application.{\it Biometrika} 1981, {\bf 68}(1): 265-274.

\bibitem{ingham_1933} Ingham, A.~E. An integral which occurs in statistics. {\it Mathematical
  Proceedings of the Cambridge Philosophical Society}, 1933, {\bf 29}(2): 271 -- 276.   %%%

\bibitem{4} Surname(s) Initial(s). {B}ook {T}itle. Site of Publication: Publishing House,
Year of Publication: relevant numbers. %% Books

%Books:作者姓(首字母大写) 名的首字母缩写. 书名(实词首字母大写).出版地: 出版商,出版年: 引用页码.

\bibitem{5} Mor\'{e} J J. Recent Development in Algorithm and Software for
Trust Region Methods. In: Bachem A, Grotschel M, Kortz B. Mathematical Programming: The State of the Art.
Berlin: Springer-Verlag, 1983.

\bibitem{6}
Conn A R, Could N I M, Toint P. L. Trust-Region Methods:
MPS-SIAM Series on Optimization. Philadelphia, PA, USA: SIAM, 2000.

\bibitem{7} Yuan Y, Sun W. Optimization Theory and Methods. Beijing: Science Press, 1997.

\bibitem{8} Zhang X S, Zhang J L, Liao L Z. An adaptive trust region
method and its convergence. {\it Sci. China}, 2002, {\bf 45}: 620--631.

\bibitem{9} Zhang J, Xu C. Trust region dogleg path algorithms for
unconstrained minimization. {\it Ann. Oper. Res.}, 1999, {\bf 87}: 407--418.

\end{thebibliography}

%%%%%%%%%%%%%%%%%%%%%end%%%%%%%%%%%%%%%%%%%%%%

\end{document}
